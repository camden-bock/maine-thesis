% !TEX root =  Main.tex
\chapter{Front.tex}

The front matter of your thesis is primarily made up of special things that are required by the Graduate School, but also contains some optional elements.  Table \ref{front} provides a summary of these elements.

\begin{table}
\begin{minipage}{\textwidth}
\centering
\begin{tabular*}{\textwidth}{l@{\extracolsep{\fill}}r}
\hline\hline
Thesis Element & Required or Optional\\
\hline
Copyright Page & Optional\\
Abstract & Required\\
Lay Abstract & Special\footnote{See section \ref{abstracts}}\\
Preface & Optional\\
Dedication & Optional\\
Acknowledgements & Optional\\
Table of Contents & Required\\
\hline\hline
\end{tabular*}\\
\end{minipage}
\caption{The elements of the front matter for your thesis.}
\label{front}
\end{table}

\section{Copyright}\label{copy}

\begin{verbatim}
\copyrightpage[copyright holder]{year}
\end{verbatim}

This command creates a copyright page.  This page is optional (unless you've taken the time to register the copyright, in which case it's required by law, not the Graduate School), so you can neglect this page if you want to.  If you do issue it, there are a pair of arguments that it takes.  The first (between ``['' and ``]'') specifies the copyright holder.  This argument can be left off completely (in which case the ``['' and ``]'' are also not necessary) and will default to you, the author.  The second argument is required and declares the year of the copyright.  If parts of your thesis were supported by grants or were previously published, you should consult with your advisor and the prior publishers to make sure that you specify these arguments correctly before including this page.

\textbf{Note:} In 2016 (v1.12 of this class file) the Graduate School switched to an electronic thesis submission process and eliminated the Dissertation Acceptance and Library Rights Statement pages which had previously surrounded the Copyright page.  If you are recompiling an older thesis which contains these pages, then you will need to use earlier version of the class file.

\section{University of Maine Graduate School Land Acknowledgment}\label{glsa}
\begin{verbatim}
\glsa{}
\end{verbatim}

This command creates the University of Maine Graduate School Land Acknowledgment. This page is optional and takes no parameters. The land acknowledgment should be after the copyright page, as shown by the thesis template made by the Graduate School. 

\section{Abstract(s)}\label{abstracts}
The graduate requires two abstracts, but only one has to appear in the bound dissertation.

This first abstract is the usual abstract you would write for a scholarly journal in your field.  This is the abstract that must be in the bound thesis.  It should be limited to 500 words for a Master's Thesis and 350 words for a Doctoral Thesis.  It cannot contain formulas, tables, diagrams, or other illustrations.  Typesetting your abstract is accomplished with the ``abstract'' environment:
\begin{verbatim}
\begin{abstract}
...
...
...
\end{abstract}
\end{verbatim}

The second abstract is one suitable for a lay audience.  This abstract is limited to 350 words regardless of the kind of degree you're getting and should not contain highly technical language.  It should be written with the expectation that the reader will have only minimal knowledge of your field as it may be ``submitted [...] for publication in newspapers, magazines, and other media of
interest to the general public, and it may be used in selecting nominees for regional and
national competitions.''  It must be accompanied by at least 5 keywords (for search engines to pick up on, presumably) and it may contain 1 (and only 1) image.  If you didn't create said image, then you need permission of the copyright holder to use it.

This abstract doesn't have to be bound with your thesis, but must be submitted electronically to crystal.burgess@maine.edu.  When submitting this abstract, a Word document that is formated correctly is preferred due to some copy/paste peculiarities between Adobe Reader and the form the grad school uses to upload the file to the web.  If you don't have access to Word, however, you can get away with a pdf version.  This pdf version can be generated by this class file with the ``layabstract'' environment; the same environment you'd use to generate the lay abstract for inclusion in the bound copy:

\begin{verbatim}
\begin{layabstract}{...}
...
...
...
\end{layabstract}
\end{verbatim}

In this case, the environment argument is the list of keywords, while the body of the abstract should be within between the environment commands.  If you do include an image, do not enclose it within a figure environment as it should not appear in the List of Figures.

\section{Dedication}
If there is some person (or group of persons) to whom you want to dedicate your thesis, then you'll need to use the dedication environment.  This should be short, and is optional:
\begin{verbatim}
\begin{dedication}
...
\end{dedication}
\end{verbatim}

\section{Preface}
If you want to include a preface to your thesis then you typeset it with the preface environment.  This can be long or short and is optional.

\begin{verbatim}
\begin{preface}
...
...
...
\end{preface}
\end{verbatim}

\paragraph{Note:}The document structure within the Preface is unnumbered (a Graduate School requirement).

\section{Acknowledgements}
While considered optional by the Graduate School, the acknowledgments are the appropriate place to mention funding sources, collaborators, and anyone who helped with the writing or revision of your thesis.  They are typeset with the acknowledgements environment:

\begin{verbatim}
\begin{acknowledgements}
...
...
...
\end{acknowledgements}
\end{verbatim}

\section{Table of Contents}
The last element of the front matter is the table of contents.  This actually consists of several lists, the first of which is actually called ``Table of Contents'' and contains the name and page numbers of chapters, sections, subsections, and chapter-like elements.  The other lists are all pseudo-optional.  If they would be populated (i.e. if you have tables or figures), then they need to be there.  If they are empty, then you can leave the empty list off.  Typesetting these lists is handled with a series of commands:

\begin{verbatim}
\tableofcontents
\listoftables
\listoffigures
\end{verbatim}

In addition, the Graduate School requires you to have other lists for ``a consistent set of theorems, symbols, abbreviations or definitions'' should such a set appear in your thesis.  Some packages add \verb=\listof*= commands to create and auto-populate the list for the element that they are support just like \verb=\listoftables= and \verb=\listoffigures= do for tables and figures.  If so, you should probably use said command as it will make your life much easier (though pay attention to the formatting that the command creates, you may need to modify it manually).  However, for those instances where the package doesn't do so, there is a ``listof'' environment which you can use to manually create such a page:

\begin{verbatim}
\begin{listof}{...}
...
...
...
\end{listof}
\end{verbatim}

\section{File Close}
The second to last line that should be in your Front file signals the start of the main body of the thesis with the command \verb=\mainmatter=.  This resets the page numbering, changes it to arabic numerals, switches to double spacing, and adds the word ``Chapter'' to your table of contents before your first chapter.

Since this isn't strictly creating a piece of the front matter of your thesis it might seem more logical to put this command in Main.tex after \verb=% !TEX root =  Main.tex
%\preparetocs
\copyrightpage[<Copyright holder appears here, see Section \ref{copy} for details>]{<Copyright year appears here>}

\begin{abstract}
The abstract for you thesis will appear on this page.  It should be limited to 350 words for a Ph.D thesis or 500 words for a Master's thesis.
\end{abstract}

\begin{layabstract}{Lay abstract keywords appear here}
The Grad School now requires a lay abstract of up to 350 words for all theses, but also requires that the lay abstract be submitted electronically to Lauren E. Dupee (email: \verb=lauren.e.dupee@maine.edu=).  Whether you include it in the bound version of your thesis is your choice.

If you do want to include the lay abstract in the bound version of your thesis then it will appear here.
\end{layabstract}

\begin{preface}
\section{Preface from the 2003 version}

 This class file is written for use with the \LaTeXe\ document preparation system for theses
 conforming to the guidelines of the Graduate School at the University of Maine.  
 Ideas for this class were found in  
 the class files \verb=gt-thesis.cls=\footnote{available at
 http://www.ctan.org} and \verb=rpithesis.cls=.\footnote{can be currently found at
  http://www.rpi.edu/computing/software/latex/thesis-info.html}  This class
 file is relatively compact, without too many options for the user.  A majority of the
 credit for this class should go to the original writers of those two classes.  

 \paragraph{What This Class File Can Do---}
 \verb=maine-thesis.cls= can format a masters or doctoral thesis according to the guidelines set
 forth by the Graduate School of the University of Maine.  It produces a double spaced, one sided
 document with the correct margins for final publication. 
 It will properly format a titlepage, 
 optional copyright page, abstract, optional dedication, acknowledgements and preface sections,
 a table of contents, lists of tables and figures, main matter, and end matter.  The Graduate
 School is relatively lenient in some formatting issues and strict in others.  Where there is
 leniency, decisions were made that I thought looked best.  Changes can be made to the class file
 to make it look more to your liking, but in its current version, this class file
 will produce a thesis that is acceptable to the Graduate School.
 \paragraph{A Final Note---}
 A word of warning:  {\bfseries THE GUIDELINES OF THE GRADUATE SCHOOL CAN AND DO CHANGE.}	
 This class was written using the most recent set of guidelines [...]\footnote{Date of the guidelines used by the original package author has been removed to avoid confusion.}.  
 They do change every so often so be sure that you have a copy of the most recent set of
 guidelines.  Most changes that are made will probably be small and cosmetic but there is no guarantee that
 something major will not arise.  
 
 \begin{flushright}
 Jim Kenneally
 \end{flushright}
 
\section{Preface to version 1.5}
As Jim predicted, the guidelines of the graduate school have changed a bit over the years since he originally designed this class file.  Over those intervening years every successive person who has used the class file has been required by the Graduate School to make some changes: some in response to things which Jim didn't get quite right, most to things which they had changed or become stricter on.  In most cases those changes accumulated in various versions of the file and were handed on to the next person interested in using the class file.  In some cases, however, a student would leave before they handed the class file on to anyone and as a result any changes they made would be lost and have to be reproduced.

I've made the effort to acquire all versions of the class file that I can and to consolidate the changes they contain into a single project.  I've also requested feedback from the Graduate School to make sure that this package conforms to their current standards.  This is the result.  I hope you find it useful and easy to use.

If the Graduate School requires you to change some aspect of your thesis formatting which you believe should be taken care of by this class file, please email me (\email) with a detailed description of the problem and a simple sample document that reproduces it (I don't want your whole thesis, just the part that's not right).  While I cannot guarantee that I will get to it right away, I will look at the problem just as soon as I have time and will endeavor to fix it.  If you can't afford to wait for me to fix the problem and find a fix that works, please email me that fix as well, as it's much easier for me to incorporate a fix than it is to diagnose and fix a problem.

As of April 2016, this thesis class is now hosted on GitHub (\url{https://github.com/rpspringuel/maine-thesis}).  If you have a GitHub account (they are free), you can submit Issues (bug reports) and Pull Requests (suggested changes) there.

\begin{flushright}
R. Padraic Springuel\\
Most recent version of Graduate School guidelines used: June 2015\\
Documentation last edited on \today
\end{flushright}

\end{preface}

\begin{dedication}
Dedications are optional, but if you have one it will appear here.
\end{dedication}

\begin{acknowledgements}
While acknowledgements are technically optional, they are also the perfect place to make note of funding sources, collaborators, and other people whose work made your thesis possible.  This is also the place to mention an External Reader (i.e. some one from outside the University who read and commented on your thesis) if you have one.  Acknowledgements appear here.
\end{acknowledgements}

\tableofcontents
\listoftables
\listoffigures

\begin{thesislist}{Whatever}
If you have some an consistent set of theorems, symbols, abbreviations, or definitions, then you must include a page which lists them just as you list the tables and figures in your thesis.
\end{thesislist}

\mainmatter

\endinput=, however, this command suffers from the same bug that effects \verb=\references=.  However, since this command comes first in the document, it appears to be subject to it more reliably.  Putting the command at the end of Front.tex dodges that bug (as would placing it at the beginning of Ch1.tex).  It's not elegant, but it works.

The last line of Front.tex is \verb=\endinput=.  This command isn't technically necessary (i.e. your document will typeset just fine without it), but it is good programming practice to include it.  If it is used, then anything that appears after it will be ignored by \LaTeX, making it a great way to create a scratch space at the end of each file where you can write notes to yourself.  You don't even have to comment them out!
\endinput

This is a note to myself which is ignored by LaTeX due to the \endinput command above it.
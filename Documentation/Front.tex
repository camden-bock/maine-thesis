% !TEX root =  Main.tex
%\preparetocs
\copyrightpage[<Copyright holder appears here, see Section \ref{copy} for details>]{<Copyright year appears here>}

\begin{abstract}
The abstract for you thesis will appear on this page.  It should be limited to 350 words for a Ph.D thesis or 500 words for a Master's thesis.
\end{abstract}

\begin{layabstract}{Lay abstract keywords appear here}
The Grad School now requires a lay abstract of up to 350 words for all theses, but also requires that the lay abstract be submitted electronically to Lauren E. Dupee (email: \verb=lauren.e.dupee@maine.edu=).  Whether you include it in the bound version of your thesis is your choice.

If you do want to include the lay abstract in the bound version of your thesis then it will appear here.
\end{layabstract}

\begin{preface}
\section{Preface from the 2003 version}

 This class file is written for use with the \LaTeXe\ document preparation system for theses
 conforming to the guidelines of the Graduate School at the University of Maine.  
 Ideas for this class were found in  
 the class files \verb=gt-thesis.cls=\footnote{available at
 http://www.ctan.org} and \verb=rpithesis.cls=.\footnote{can be currently found at
  http://www.rpi.edu/computing/software/latex/thesis-info.html}  This class
 file is relatively compact, without too many options for the user.  A majority of the
 credit for this class should go to the original writers of those two classes.  

 \paragraph{What This Class File Can Do---}
 \verb=maine-thesis.cls= can format a masters or doctoral thesis according to the guidelines set
 forth by the Graduate School of the University of Maine.  It produces a double spaced, one sided
 document with the correct margins for final publication. 
 It will properly format a titlepage, 
 optional copyright page, abstract, optional dedication, acknowledgements and preface sections,
 a table of contents, lists of tables and figures, main matter, and end matter.  The Graduate
 School is relatively lenient in some formatting issues and strict in others.  Where there is
 leniency, decisions were made that I thought looked best.  Changes can be made to the class file
 to make it look more to your liking, but in its current version, this class file
 will produce a thesis that is acceptable to the Graduate School.
 \paragraph{A Final Note---}
 A word of warning:  {\bfseries THE GUIDELINES OF THE GRADUATE SCHOOL CAN AND DO CHANGE.}	
 This class was written using the most recent set of guidelines [...]\footnote{Date of the guidelines used by the original package author has been removed to avoid confusion.}.  
 They do change every so often so be sure that you have a copy of the most recent set of
 guidelines.  Most changes that are made will probably be small and cosmetic but there is no guarantee that
 something major will not arise.  
 
 \begin{flushright}
 Jim Kenneally
 \end{flushright}
 
\section{Preface to version 1.5}
As Jim predicted, the guidelines of the graduate school have changed a bit over the years since he originally designed this class file.  Over those intervening years every successive person who has used the class file has been required by the Graduate School to make some changes: some in response to things which Jim didn't get quite right, most to things which they had changed or become stricter on.  In most cases those changes accumulated in various versions of the file and were handed on to the next person interested in using the class file.  In some cases, however, a student would leave before they handed the class file on to anyone and as a result any changes they made would be lost and have to be reproduced.

I've made the effort to acquire all versions of the class file that I can and to consolidate the changes they contain into a single project.  I've also requested feedback from the Graduate School to make sure that this package conforms to their current standards.  This is the result.  I hope you find it useful and easy to use.

If the Graduate School requires you to change some aspect of your thesis formatting which you believe should be taken care of by this class file, please email me (\email) with a detailed description of the problem and a simple sample document that reproduces it (I don't want your whole thesis, just the part that's not right).  While I cannot guarantee that I will get to it right away, I will look at the problem just as soon as I have time and will endeavor to fix it.  If you can't afford to wait for me to fix the problem and find a fix that works, please email me that fix as well, as it's much easier for me to incorporate a fix than it is to diagnose and fix a problem.

\section{Preface to version 1.9}
As of April 2016, this thesis class is now hosted on GitHub (\url{https://github.com/rpspringuel/maine-thesis}).  If you have a GitHub account (they are free), you can submit Issues (bug reports) and Pull Requests (suggested changes) there.

\begin{flushright}
R. Padraic Springuel\\
Most recent version of Graduate School guidelines used: March 2019\\
Documentation last edited on \today
\end{flushright}

\end{preface}

\begin{dedication}
Dedications are optional, but if you have one it will appear here.
\end{dedication}

\begin{acknowledgements}
While acknowledgements are technically optional, they are also the perfect place to make note of funding sources, collaborators, and other people whose work made your thesis possible.  This is also the place to mention an External Reader (i.e. some one from outside the University who read and commented on your thesis) if you have one.  Acknowledgements appear here.
\end{acknowledgements}

\tableofcontents
\listoftables
\listoffigures

\begin{thesislist}{Whatever}
If you have some an consistent set of theorems, symbols, abbreviations, or definitions, then you must include a page which lists them just as you list the tables and figures in your thesis.
\end{thesislist}

\mainmatter

\endinput
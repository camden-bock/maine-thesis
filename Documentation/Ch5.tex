% !TEX root =  Main.tex
\chapter{Other Stuff}
\section{\textbackslash ignore}
The class file defines the command \verb=\ignore{...}= which is very useful for removing large blocks of text from the thesis without deleting them.  Anything within the argument is treated as if it was commented out and will not appear in the typeset document.

\section{\textbackslash comment}
There is also a \verb=\comment{...}= command.  If the draft option was issued to the class file, the argument of this command will appear in the right hand margin in red with ``NOTE:'' preceding it in a smaller font (as can be seen in this file if you typeset it in draft mode).  This makes it useful for adding reminders to yourself about things you still need to do, or questions for your advisor when you're asking him to review a draft version of something.  Without the draft option this command functions identically to \verb=\ignore=.\comment{There is a comment here.}

\section{\textbackslash highlight}
The \verb=\highlight= command is a useful companion to \verb=\comment=.  In draft mode it will highlight its argument (i.e., give it a yellow background).  In final mode, the text appears normally.

\section{\textbackslash pocket}
If you have supplementary materials such as a DVD or CD which will be stored in a pocket inside the cover of your thesis the Graduate school requires you to list these in the table of contents.  This can be done with the \verb=\pocket{...}= command.  However, like the \verb=\references= command, the entry in the table of the contents that this produces may be out of place if it is put in Main.tex.  See Section \ref{bib} for more details.

Table of contents entries created by this command have the status of chapters (or appendicies, if you've already issued the \verb=\appendix= command) and will increment the appropriate counter.

\section{\textbackslash toclabel}
If you need to add a label into your table of contents then you can use the command \verb=\toclabel{...}= to do so.  Due to a bug in the \LaTeX\ kernel that effects this command you may need to put the command inside a chapter file rather than in Main.tex. Otherwise the label may not end up in the correct place in the TOC.

\section{\textbackslash compresstitlepage}
In rare cases your title page may spill over onto a second page when typeset with double line spacing (generally due to a long title, many previous degrees, or committee members with long/multiple titles).  When this happens, issue the command \verb=\compresstitlepage= in your preamble.  This will change the line spacing for the committee members to single line spacing.  If that isn't enough to get your title page onto one page, add the optional argument (``[2]'') to change the spacing for the rest of the title page to one-and-a-half spacing.\comment{There is another comment here.}

\section{verbatim and \textbackslash verb}
Since the Graduate School doesn't want the font to change during the course of the document, the verbatim environment and the \verb=\verb= command have had their font changed from the standard typewriter font of \LaTeX\ to the normal roman font.  This required a change to the font encoding to get what you type in the verbatim environment to be the same characters that appear on the page.  As a result, the mapping of quotation marks, ", to close double quotes, '', doesn't work.  To get '' you will need to type two close quotation marks, '\,', just like you have to type two two open quotation marks, `\,`, to get the double open quotes, ``.

\section{Widows and Clubs}
The graduate school requires that page breaks occur so that at least 3 lines of any paragraph are on a page.  Thus, if a paragraph starts on a certain page at least the first three lines of that paragraph should be on that page.  Likewise, if a paragraph ends on a page at least the last three lines of that paragraph should be on that page.  This is an unusually stringent requirement for clubs (paragraph starts at the end of a page) and widows (paragraph ends at the beginning of a page) and one which is impossible to force \LaTeX\ to respect.  The club penalty and widow penalty have been set high to large values so that at least two lines of each paragraph should appear on each page, but if the graduate school starts bugging you about this, you are going to have to play with this manually using \verb=\pagebreak= and \verb=\nopagebreak=.  Do this only when preparing the \textit{absolutely final copy} of the thesis as said manual breaks will stick around despite any subsequent edits to the document.

Likewise, the Graduate School has similar standards for the table of contents: at least 3 entries from any given chapter must be on the page (unless the chapter has fewer than 3 entries, in which case all entries should appear on the same page).  I've done my best to make sure this happens, but I can't possibly test every possible pattern of chapters, sections, and subsections that you might have.  If the graduate school is bugging you about this, then place \verb=\addtocontents{toc}{\protect\pagebreak}= just before a chapter, section, or subsection to manually insert a page break into that position in your table of contents.  This command can likewise be used to manually insert page breaks into the list of figures (lof) or the list of tables (lot) by changing the first argument.  It also will run into the same bug that effects \verb=\references=.  See Section \ref{bib} for more details.  If you do have to do this, I'd also appreciate a minimal working example so that I can try to further fine tune the class file's ability to do this automatically.

\section{Thesis in a Foreign Language}
The class file has not been tested on a thesis written in a foreign language and thus its behavior on such documents is not guaranteed.  Support for these kinds of documents is planned for a future version, but probably won't come until 2019 at the earliest.  Contributions designed to make the class file work with foreign language theses are appreciated: \verb=R.Springuel@umit.maine.edu=.

\section{Hyphenation \& Justification}
At the graduate school request, automatic hyphenation is turned off and the document should be set left justified (\verb=\raggedright= in LaTeX parlance).  If this creates strange behavior for you, please let me know so that any possible bugs can be resolved.

\section{5-dot Leader Minimum in TOC}
As of v1.10 the this requirement should be obeyed automatically.  If you run into problems here please report it to me (\email).  As a work around, you can fix this manually by either changing the appropriate title/caption, or by making use of the optional short title/caption built into the appropriate command.

\endinput

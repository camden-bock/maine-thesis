% maine-thesis.dtx
%<*class>

\ProvidesClass{maine-thesis}[2025/09/01 v2.1 Maine thesis class]

\NeedsTeXFormat{LaTeX2e}[1999/12/01]

%%%%%%%%%%%%%%%%%%%%%
% ------INITIAL CODE------
%%%%%%%%%%%%%%%%%%%%%
\newif\ifdraft\draftfalse
\newif\iftwoside\twosidefalse
\newif\ifmt@official\mt@officialtrue
\newif\ifunbound\unboundfalse
\newif\ifmt@legacycaptions\mt@legacycaptionsfalse
\newif\ifmt@loftspace\mt@loftspacefalse
\newcommand\docsize{}
\newcommand\side{oneside}
\newcounter{secnumdefault}
\setcounter{secnumdefault}{3}
\newcounter{head} %Heading styles get numbered so that \ifcase can be used when determining which heading style is currently set instead of a nested set of \if and \else.
\setcounter{head}{1000} %The default value is set very large so that adding new styles shouldn't necessitate changing it.
\def\@margg{1.5in}
\def\mt@pagestyle{bottom}
%%%%%%%%%%%%%%%%%%%%%
% ------DECLARATION OF OPTIONS------
%%%%%%%%%%%%%%%%%%%%%

\DeclareOption{10pt}{\renewcommand\docsize{10pt}}
\DeclareOption{11pt}{\renewcommand\docsize{11pt}}
\DeclareOption{12pt}{\renewcommand\docsize{12pt}}
\DeclareOption{draft}{\drafttrue
    \ExecuteOptions{10pt}
    \PassOptionsToClass{draft}{report}
    \newcommand{\drafttext}{\sffamily\small{DRAFT}
    \mt@officialfalse}
}
\DeclareOption{twoside}{\mt@officialfalse\unboundfalse\twosidetrue\renewcommand\side{twoside}\def\@margg{1.5in}}
\DeclareOption{oneside}{\mt@officialfalse\unboundfalse\renewcommand\side{oneside}\def\@margg{1.5in}}
\DeclareOption{unbound}{\mt@officialtrue\unboundtrue\def\@margg{1in}}

\def\HeadingStyle{} % Define an empty macro as a placeholder
\DeclareOption{apa}{
    \setcounter{secnumdefault}{3}
    \setcounter{head}{0}
    \def\HeadingStyle{apa}}
\DeclareOption{chicago}{
    \setcounter{secnumdefault}{3}
    \setcounter{head}{1}
    \def\HeadingStyle{chicago}}
\DeclareOption{headings}{
    \setcounter{secnumdefault}{3}
    \setcounter{head}{2}
    \def\HeadingStyle{mainethesis}}
\DeclareOption{idecimal}{
    \setcounter{secnumdefault}{3}
    \setcounter{head}{3}
    \def\HeadingStyle{idecimal}}
\DeclareOption{jdecimal}{
    \setcounter{secnumdefault}{3}
    \setcounter{head}{1000}
    \def\HeadingStyle{jdecimal}}
\DeclareOption{legacycaptions}{\mt@legacycaptionstrue}
\DeclareOption{pagenumberbottom}{\def\mt@pagestyle{bottom}}
\DeclareOption{pagenumbertop}{\def\mt@pagestyle{top}}
\DeclareOption*{\PassOptionsToClass{\CurrentOption}{report}}
\DeclareOption{loftspacing}{\mt@loftspacetrue\mt@officialfalse}

% Define a macro to store the citation style
\newcommand*{\@citestyle}{authoryear}

% Declare the 'citestyle' option
% #1 represents the argument passed to the option
\DeclareOption{citestyle}{%
  \renewcommand*{\@citestyle}{#1}%
}

% Set a default value for 'citestyle'
\ExecuteOptions{citestyle=authoryear}

%%%%%%%%%%%%%%%%%%%%%
% ------EXECUTION OF OPTIONS------
%%%%%%%%%%%%%%%%%%%%%
\ExecuteOptions{12pt,unbound}
\ProcessOptions

%%%%%%%%%%%%%%%%%%%%%
% ------PACKAGE LOADING------
%%%%%%%%%%%%%%%%%%%%%

\LoadClass[\docsize,\side]{report}[2007/10/19 v1.4h Standard LaTeX document class]
\RequirePackage[T1]{fontenc}[2005/09/27 v1.99g Standard LaTeX package]
\RequirePackage{fontspec}
\RequirePackage{fontspec}
\RequirePackage{microtype} % Optional, but recommended for better typesetting
\RequirePackage{color}[2005/11/14 v1.0j Standard LaTeX Color (DPC)]
\RequirePackage[norule]{footmisc}[2009/09/15 v5.5a a miscellany of footnote facilities]
\RequirePackage{soul}[2003/11/17 v2.4 letterspacing/underlining  (mf)]
\RequirePackage{afterpackage}[2006/01/17 v1.1 Apply Commands After Package (NCC)]
\RequirePackage{etoolbox}[2015/08/02 v2.2a e-TeX tools for LaTeX (JAW)]
\RequirePackage[none]{hyphenat}[2009/09/02 v2.3c hyphenation utilities]
\RequirePackage{iftex}[2013/04/04 v0.2 Provides if(tex) conditional for PDFTeX, XeTeX, and LuaTeX]
\RequirePackage{geometry}[2010/09/12 v5.6 Page Geometry]
\RequirePackage{xstring}[2013/10/13 v1.7c  String manipulations (C Tellechea)]
\RequirePackage{textcase} %% safe manipulation of text case
\RequirePackage{titlesec} %% safer heading generation
\RequirePackage{setspace} %% single, double and oneandahalf spacing -- following same parameters, are default in this package
\RequirePackage{titling} % title page generation
\RequirePackage[backend=biber, style=\@citestyle,  doi = true, autopunct = true, date=year]{biblatex} %biblatex citations

%% Additional  Package Inclusions
\RequirePackage{lscape} %adds ability to rotate pages
\RequirePackage[titles]{tocloft} %Ttitle page generation
\usepackage[nottoc]{tocbibind} %adds list of tables and list of figures to the table of contents
\RequirePackage[utf8]{inputenc} %allows latex to handle unicode characters
\RequirePackage{textcomp} %Adds trademark TM
\RequirePackage[super]{nth} %adds counting of numbers with *th
\RequirePackage{datetime}
\RequirePackage{csquotes}

%Formatting numbers, special characters and chemical formulas
\RequirePackage[version=4]{mhchem} %adds chemical formulas
\RequirePackage{textgreek}
\RequirePackage{siunitx} %adds SI unit compatibility

%Formatting the references
\RequirePackage{hyperref}

%Formatting the figures
\RequirePackage{graphicx} %Allows us to have graphics embedded
\RequirePackage{rotating} %Allows graphics to be rotated and tables
\RequirePackage[margin=10pt,font=small,labelfont=bf,labelsep=endash]{caption}  %adds a caption to the figure

%Formatting tables
\RequirePackage{float}
\RequirePackage{booktabs}
\RequirePackage[table,xcdraw]{xcolor}
\RequirePackage{multirow}
\RequirePackage{lscape}
\RequirePackage{longtable}

%% Draft Annotations
\RequirePackage{lineno}

%Formatting comments that will be printed in the margins of the compiled pdf
\RequirePackage[colorinlistoftodos]{todonotes}

\PassOptionsToPackage{caption=false}{subfig}

%%% ------------------------------------------------------------------
%			Module 2: Custom Commands
%%% ------------------------------------------------------------------
% TODO: ideally, we want to reduce these and use our dependencies insetead,
% but still provide backwards compatability

\newcommand{\comment}[1]{
    \ifdraft
    \marginpar{
        \color{red}\flushleft\scriptsize\setlength{\baselineskip}{7pt}
        {\MakeTextUppercase{Note}: #1}}
    \fi
}

\newcommand{\ignore}[1]{}

\newcommand{\highlight}[1]{%
    \ifdraft%
        \hl{#1}%
    \else%
        #1%
    \fi%
}
%%% ------------------------------------------------------------------
%           Module A: General Formatting
%%% -------------------------------------------------------------------

\ifdraft{\linenumbers}\fi

\setmainfont{EB Garamond}

\geometry{
    letterpaper,
    margin=1in,
    left=\@margg,
    headsep=0.4in,
    headheight=14pt,
    footskip=30pt,
    marginparwidth=40pt,
    marginparsep=10pt}

\setcounter{secnumdepth}{\value{secnumdefault}}
\setcounter{tocdepth}{\value{secnumdefault}}
\raggedbottom
\raggedright
\parindent=1.5em\relax
\markboth{}{}
\clubpenalty=10000
\widowpenalty=10000

\def\verbatim@font{\rmfamily}

\newcommand*\loftspacing{10}

%%%%%%%%%%%%%%%%%%%%%
% Header Formatting Declarations
%%%%%%%%%%%%%%%%%%%%%

%Macro for period after heading
\def\adddot#1{#1.\thinspace}
%macro for underline
\def\addul#1{\underline{#1}}
\def\adddotul#1{\underline{#1}.\thinspace}

\newcommand{\apaheadings}{%
    \typeout{applying apa headings}
    %--- APA 7 Style Headings ---
    \titleformat{\chapter}[display]{\normalfont\centering\bfseries}{\chaptertitlename\ \thechapter}{-3pt}{\MakeTextUppercase}
    \titlespacing{\chapter}{0pt}{50pt}{15pt}
    \titleformat{\section}{\normalfont\bfseries}{\thesection}{0em}{}
    \titlespacing{\section}{0pt}{3.5ex plus 1ex minus .2ex}{2.3ex plus .2ex}
    \titleformat{\subsection}{\normalfont\bfseries\itshape}{\thesubsection}{0em}{}
    \titlespacing{\subsection}{0pt}{3.25ex plus 1ex minus .2ex}{1.5ex plus .1ex}
    \titleformat{\subsubsection}[runin]{\normalfont\bfseries}{\thesubsubsection}{1em}{\adddot}
    \titlespacing{\subsubsection}{\parindent}{3.25ex plus 1ex minus .2ex}{0pt}[1em]
    \titleformat{\paragraph}[runin]{\normalfont\bfseries\itshape}{\theparagraph}{1em}{\adddot}
    \titlespacing{\paragraph}{\parindent}{3.25ex plus 1ex minus .2ex}{0pt}[1em]
}

\newcommand{\chicagoheadings}{%
    % Level 1: Centered, Boldface, Headline-style
    \titleformat{\chapter}[display]{\normalfont\centering\bfseries}{\chaptertitlename\ \thechapter}{-3pt}{\MakeTextUppercase}
    \titlespacing{\chapter}{0pt}{50pt}{15pt}

    % Level 2: Centered, Regular, Headline-style
    \titleformat{\section}{\normalfont\centering\normalsize}{\thesection}{1.5ex}{\MakeTextUppercase}
    \titlespacing{\section}{0pt}{3.5ex plus 1ex minus .2ex}{2.3ex plus .2ex}

    % Level 3: Flush Left, Bold Italic, Headline-style
    \titleformat{\subsection}{\normalfont\bfseries\itshape\normalsize}{\thesubsection}{0em}{\MakeTextUppercase}
    \titlespacing{\subsection}{0pt}{3.25ex plus 1ex minus .2ex}{1.5ex plus .1ex}

    % Level 4: Flush left, Roman type, Sentence-style
    \titleformat{\subsubsection}{\normalfont\normalsize}{\thesubsubsection}{0em}{}
    \titlespacing{\subsubsection}{0pt}{3.25ex plus 1ex minus .2ex}{0pt plus .1ex}

    % Level 5: Run in, Bold Italic, Sentence-style, with period
    \titleformat{\paragraph}[runin]{\normalfont\bfseries\itshape}{\theparagraph}{1em}{\adddot}
    \titlespacing{\paragraph}{\parindent}{3.25ex plus 1ex minus .2ex}{0pt}[1em]
}

\newcommand{\mainethesisheadings}{%
    %--- Sample Headings from Guidelines ---
    % Chapter Titles
    \titleformat{\chapter}[display]{\normalfont\centering\bfseries}{\chaptertitlename\ \thechapter}{-3pt}{\MakeTextUppercase}
    \titlespacing{\chapter}{0pt}{50pt}{15pt}

    % First Level Headings
    \titleformat{\section}{\normalfont\bfseries}{}{0pt}{\addul}
    \titlespacing{\section}{0pt}{3.5ex plus 1ex minus .2ex}{2.3ex plus .2ex}

    % Second Level Headings
    \titleformat{\subsection}{\normalfont\bfseries}{}{0pt}{}
    \titlespacing{\subsection}{0pt}{3.25ex plus 1ex minus .2ex}{1.5ex plus .1ex}

    % Third Level Headings
    \titleformat{\subsubsection}[runin]{\normalfont\bfseries}{}{0pt}{\adddotul}
    \titlespacing{\subsubsection}{0pt}{3.25ex plus 1ex minus .2ex}{0pt}

    % Fourth Level Headings
    \titleformat{\paragraph}[runin]{\normalfont}{}{0pt}{\adddotul}
    \titlespacing{\paragraph}{0pt}{3.25ex plus 1ex minus .2ex}{0pt}
}

\newcommand{\idecimalheadings}{%
    \setcounter{tocdepth}{4}
    % Chapter Titles: Centered and bold
    \titleformat{\chapter}[display]{\normalfont\centering\bfseries}{\thechapter}{1em}{}
    \titlespacing{\chapter}{0pt}{50pt}{15pt}

    % Level 1 Headings: Bold, followed by a line break
    \titleformat{\section}{\normalfont\bfseries}{\thesection}{1em}{}
    \titlespacing{\section}{0pt}{3.5ex plus 1ex minus .2ex}{2.3ex plus .2ex}

    % Level 2 Headings: Bold, followed by a line break
    \titleformat{\subsection}{\normalfont\bfseries}{\thesubsection}{1em}{}
    \titlespacing{\subsection}{0pt}{3.25ex plus 1ex minus .2ex}{1.5ex plus .1ex}

    % Level 3 Headings: Bold, run-in with a period, with 1em spacing between number and title
    \titleformat{\subsubsection}[runin]{\normalfont\bfseries}{\thesubsubsection}{1em}{\adddot}
    \titlespacing{\subsubsection}{\parindent}{3.25ex plus 1ex minus .2ex}{0pt}[1em]

    % Level 4 Headings: Bold, run-in with a period, with 1em spacing
    \titleformat{\paragraph}[runin]{\normalfont\bfseries}{\theparagraph}{1em}{\adddot}
    \titlespacing{\paragraph}{\parindent}{3.25ex plus 1ex minus .2ex}{0pt}[1em]
}

\newcommand{\jdecimalheadings}{%
    %--- Justified Decimal System (Default) ---
    \titleformat{\chapter}[display]{\normalfont\centering\bfseries\normalsize}{\chaptername\ \thechapter}{-3pt}{\MakeUppercase}
    \titlespacing{\chapter}{0pt}{0pt}{15pt}
    \titleformat{\section}{\normalfont\bfseries}{\thesection}{1.5ex}{\MakeTextUppercase}
    \titlespacing{\section}{0pt}{1.5ex plus .2ex minus 0pt}{.3ex plus .2ex}
    \titleformat{\subsection}{\normalfont\bfseries}{\thesubsection}{0.3ex}{}
    \titlespacing{\subsection}{1.5em}{0.3ex plus .2ex minus 0pt}{.2ex plus .1ex}
    \titleformat{\subsubsection}{\normalfont\bfseries}{\thesubsubsection}{0.2ex}{}
    \titlespacing{\subsubsection}{3.8em}{0.2ex plus .1ex minus 0pt}{0.2ex plus .1ex}
    \titleformat{\paragraph}{\normalfont\bfseries}{\theparagraph}{0pt}{}
    \titlespacing{\paragraph}{7.0em}{3.25ex plus 1ex minus .2ex}{3.25ex plus 1ex minus .2ex}
    \titleformat{\subparagraph}{\normalfont\hspace{1.5em}}{\thesubparagraph}{0pt}{\addul}
    \titlespacing{\subparagraph}{7.0em}{3.25ex plus 1ex minus .2ex}{3.25ex plus 1ex minus .2ex}
}

%%% ------------------------------------------------------------------
%           Module B: Headings Formatting
%%% -------------------------------------------------------------------

\typeout{=== Checking HeadingStyle Flag ===}

\ifdefined\HeadingStyle
  \typeout{--- Using \string\csname to execute style \HeadingStyle ---}
  \csname\HeadingStyle headings\endcsname
\else
  \typeout{--- No HeadingStyle option was selected. ---}
\fi

\typeout{=== Finished HeadingStyle Check ===}
%%% ------------------------------------------------------------------
%           Module C: Float Formatting
%%% -------------------------------------------------------------------

\let\latex@xfloat=\@xfloat
\def\@xfloat#1[#2]{%
    \latex@xfloat#1[#2]
    \centering
    \def\baselinestretch{1}\@normalsize
    \normalsize
}

\long\def\@makecaption#1#2{%
    \vskip\abovecaptionskip
    \sbox\@tempboxa{#1.~~#2}%
    \ifdim \wd\@tempboxa >\hsize
        {#1.~~#2}\par%
    \else
        \global \@minipagefalse
        \hb@xt@\hsize{\hfil\box\@tempboxa\hfil}%
    \fi
    \vskip\belowcaptionskip
}

\long\def\@caption#1[#2]#3{%
    \ifmt@legacycaptions%
        \def\mt@captiontext{#3}%.
    \else%
        \IfStrEq{#2}{#3}{\def\mt@captiontext{#3}}{\def\mt@captiontext{#2\ \ignorespaces#3}}%
    \fi%
    \addcontentsline{\csname ext@#1\endcsname}{#1}  {\protect\numberline{\csname fnum@#1\endcsname}{\ignorespaces #2}}%
    \begingroup
    \@parboxrestore
    \normalsize
    \centering
    \@makecaption{\csname fnum@#1\endcsname}{\ignorespaces\mt@captiontext}\par
    \endgroup
}

\long\def\@footnotetext#1{%
    \insert\footins{%
        \def\baselinestretch {1}
        \reset@font\footnotesize
        \interlinepenalty\interfootnotelinepenalty
        \splittopskip\footnotesep
        \splitmaxdepth \dp\strutbox \floatingpenalty \@MM
        \hsize\columnwidth
        \@parboxrestore
        \protected@edef\@currentlabel{%
            \csname p@footnote\endcsname\@thefnmark}%
        \color@begingroup
        \@makefntext{\rule\z@\footnotesep\ignorespaces#1\@finalstrut\strutbox}%
    \color@endgroup}
}

\long\def\@mpfootnotetext#1{%
    \global\setbox\@mpfootins\vbox{%
        \unvbox \@mpfootins
        \def\baselinestretch {1}
        \reset@font\footnotesize
        \hsize\columnwidth
        \@parboxrestore
        \protected@edef\@currentlabel{%
            \csname p@mpfootnote\endcsname\@thefnmark}%
        \color@begingroup
        \@makefntext{\rule\z@\footnotesep\ignorespaces#1\@finalstrut\strutbox}%
    \color@endgroup}
}
%%% ------------------------------------------------------------------
%           Module D: TOC Formatting
%%% -------------------------------------------------------------------

\renewcommand{\cftchapfont}{\normalfont\bfseries}
\renewcommand{\cftchappagefont}{\normalfont}
\renewcommand{\cftloftitlefont}{\normalfont\bfseries}

\newenvironment{thesislist}[1]{%
    \chapter*{\listname\ #1}
    \addcontentsline{toc}{chapter}{{\texorpdfstring{\MakeTextUppercase{\listname\ #1}}{\listname\ #1}}}
    \begingroup
}{\par\endgroup}

%% Define Leader Dots
\renewcommand{\cftchapleader}{\cftdotfill{\cftdotsep}}
\renewcommand{\cftsecleader}{\cftdotfill{\cftdotsep}}
\renewcommand{\cftsubsecleader}{\cftdotfill{\cftdotsep}}
\renewcommand{\cftfigleader}{\cftdotfill{\cftdotsep}}
\renewcommand{\cfttableader}{\cftdotfill{\cftdotsep}}

\setlength{\cftsecindent}{1.5em}
\setlength{\cftsecnumwidth}{2em}

%% TODO: what is bibalign??
\newcommand{\bibalign}{}

%%% ------------------------------------------------------------------
%           Module E: Titlepage Formatting
%%% -------------------------------------------------------------------

\pretitle{
    \begin{center}

    \bfseries\MakeTextUppercase
}
\posttitle{
    \end{center}
}

\preauthor{
    \begin{center}
    By \\
    \MakeUppercase
}
\postauthor{
    \\
    \@degreesheld
    \end{center}
}

\renewcommand{\maketitlehooka}{
    \thispagestyle{empty}
    \ifverybigtitlepage
        \onehalfspacing
    \else
        \doublespacing
    \fi
}
\renewcommand{\maketitlehookd}{
    \begin{center}
        A \MakeTextUppercase{\@type}\\[4pt]
        Submitted in Partial Fulfillment of the\\
        Requirements for the Degree of\\
        \expandafter{\@degree}\\
        (in \expandafter{\@program})\\
        \vskip 0.5in
        The Graduate School\\
        The University of Maine\\
        \expandafter{\@submitdate}
        \vfill
    \end{center}
    \ifbigtitlepage
        \singlespacing
    \fi
    \begin{flushleft}
    Advisory Committee:
    \begin{list}{}{%
        \setlength{\itemsep}{0pt}%
        \setlength{\topsep}{0in}%
        \setlength{\partopsep}{0pt}%
        \setlength{\itemindent}{-\parindent}%
        \setlength{\leftmargin}{1cm}%
    }
        % Check if the principal advisor is defined before printing
        \ifx\@principaladvisor\empty\else
            \item{\@principaladvisor, %
                \ifnum\numadv=2%
                    Co-%
                \fi%
                Advisor}
        \fi
        % Check for second advisor before printing
        \ifx\@secondadvisor\empty\else
            \item{\@secondadvisor, Co-Advisor}
        \fi
        % Check for each reader before printing
        \ifx\@firstreader\empty\else
            \item{\@firstreader}
        \fi
        \ifx\@secondreader\empty\else
            \item{\@secondreader}
        \fi
        \ifx\@thirdreader\empty\else
            \item{\@thirdreader}
        \fi
        \ifx\@fourthreader\empty\else
            \item{\@fourthreader}
        \fi
        \ifx\@fifthreader\empty\else
            \item{\@fifthreader}
        \fi
    \end{list}
    \end{flushleft}
    \clearpage
}
%%%%%%%%%%%%%%%%%%%%%%%%%%
%% Abstract             %%
%%%%%%%%%%%%%%%%%%%%%%%%%%

\newcounter{mt@page}
\renewenvironment{abstract}{%
    \doublespacing
    \begin{center}
        {\bfseries\MakeTextUppercase{\@title}}\\
        By\space\@author\\[4pt]
        \@type\ %
        \ifnum\numadv=2%
            Co-Advisors: \@principalshort and \@secondadvisor
        \else
            Advisor: \@principalshort
        \fi
        \vskip 0.33in
        \singlespacing
        An Abstract of the \@type\ Presented\\
        in Partial Fulfillment of the Requirements for the\\
        Degree of \@degree\\
        (in \@program)\\
        \@submitdate
        \vskip 36pt plus 2pt minus 12pt
    \end{center}
    \doublespacing
    \begingroup\par
    \pagestyle{empty}
}
{\cleardoublepage
    \par\endgroup
}

\newenvironment{layabstract}[1]{%
    \doublespacing
    \begin{center}
        {\bfseries\MakeTextUppercase{\@title}}\\
        By\space\@author\\[4pt]
        \@type\ %
        \ifnum\numadv=2%
            Co-%
        \fi%
        Advisor%
        \ifnum\numadv=2%
            s%
        \fi%
        : \@principalshort
        \vskip 0.33in
        \singlespacing
        A Lay Abstract of the \@type\ Presented\\
        in Partial Fulfillment of the Requirements for the\\
        Degree of \@degree\\
        (in \@program)\\
        \@submitdate
        \vskip 26pt plus 2pt minus 12pt
    \end{center}
    Keywords: \MakeLowercase{#1}
    \vskip 26pt plus 2pt minus 12pt
    \doublespacing
    \begingroup\par
    \pagestyle{empty}
}
{\cleardoublepage
    \par\endgroup
}%%%%%%%%%%%%%%%%%%%%%%%%%%
%% Acknowledgements     %%
%%%%%%%%%%%%%%%%%%%%%%%%%%

\newenvironment{acknowledgements}{%
    \chapter*{\acknowledgename}
    \addcontentsline{toc}{chapter}{\texorpdfstring{\MakeTextUppercase{\acknowledgename}}{\acknowledgename}}
    \doublespacing
    \begingroup
}{\par\endgroup}

\newenvironment{landacknowledgements}{%
    \chapter*{\landacknowledgename}
    \addcontentsline{toc}{chapter}{\texorpdfstring{\MakeTextUppercase{\landacknowledgename}}{\landacknowledgename}}
    \doublespacing
    \begingroup
}{\par\endgroup}

\newcommand{\glsa}{%
     \chapter*{UNIVERSITY OF MAINE GRADUATE SCHOOL LAND ACKNOWLEDGMENT}
     The University of Maine recognizes that it is located on Marsh Island in the homeland of Penobscot people, where issues of water and territorial rights, and encroachment upon sacred sites, are ongoing. Penobscot homeland is connected to the other Wabanaki Tribal Nations— the Passamaquoddy, Maliseet, and Micmac—through kinship, alliances, and diplomacy. The University also recognizes that the Penobscot Nation and the other Wabanaki Tribal Nations are distinct, sovereign, legal and political entities with their own powers of self-governance and self-determination.
     \vfill
     \clearpage
}
%%%%%%%%%%%%%%%%%%%%%%%%%%
%% Appendix             %%
%%%%%%%%%%%%%%%%%%%%%%%%%%

\renewcommand{\appendix}{%
    \cleardoublepage
    \doublespacing
    \setcounter{tocdepth}{1} % Sets the ToC depth to chapters and above
    \setcounter{secnumdepth}{1} % Sets the section numbering depth to chapters and above
    \setcounter{chapter}{0}
    \apptrue
    \gdef\chaptername{Appendix} % This changes the name globally
    \gdef\thechapter{\@Alph\c@chapter}% Changes the numbering globally
    \ifmultipleappendices%
        \cftaddtitleline{toc}{chapter}{\texorpdfstring{\MakeUppercase{Appendices}}{Appendices}}
    \fi
}

\pretocmd{\chapter}{\ifapp\renewcommand*\chaptername{Appendix}\fi}{}{}
%%%%%%%%%%%%%%%%%%%%%%%%%%
%% Biography             %%
%%%%%%%%%%%%%%%%%%%%%%%%%%

\newenvironment{biography}{%
    \chapter*{\bioname}
    \addtocontents{toc}{\protect\nopagebreak}
    \addcontentsline{toc}{chapter}{\texorpdfstring{\MakeTextUppercase{\bioname}}{\bioname}}
    \begingroup
    \doublespacing
    \thispagestyle{plain}
}{%
    \par
    \ifx\@authorpronoun\empty
        \@author
    \else
        \@authorpronoun
    \fi%
    \ is a candidate for the \@degree\ degree in \@program\ from the University of Maine in \@submitdate.\par\endgroup
}
%%%%%%%%%%%%%%%%%%%%%%%%%%
%% Copyrgiht             %%
%%%%%%%%%%%%%%%%%%%%%%%%%%

\newcommand{\copyrightpage}{%
    \setcounter{page}{3} % we must manually set the copyright page because of the runtime call of preliminary.
    \copyrightpagetrue
    \onehalfspacing
    \thispagestyle{plain}
    \hbox{ }
    \vfill
    \begin{center}
    \copyright \the\year{} \space \@author\\
    All Rights Reserved
    \end{center}
    \vfill
    \clearpage
}
%%%%%%%%%%%%%%%%%%%%%%%%%%
%% Dedication             %%
%%%%%%%%%%%%%%%%%%%%%%%%%%

\newenvironment{dedication}{%
    \chapter*{\dedicationname}
    \addcontentsline{toc}{chapter}{\texorpdfstring{\MakeTextUppercase{\dedicationname}}{\dedicationname}}
    \vskip 0.5in
    \doublespacing
    \begingroup
    \begin{center}
}{\end{center}\par\endgroup}
%%%%%%%%%%%%%%%%%%%%%%%%%%
%% Preface             %%
%%%%%%%%%%%%%%%%%%%%%%%%%%

\newenvironment{preface}{%
    \chapter*{\prefacename}
    \addcontentsline{toc}{chapter}{\texorpdfstring{\MakeTextUppercase{\prefacename}}{\prefacename}}
    \doublespacing
    \begingroup\setcounter{secnumdepth}{0}
}{\setcounter{secnumdepth}{\value{secnumdefault}}\par\endgroup}
%%%%%%%%%%%%%%%%%%%%%%%%%%
%% References           %%
%%%%%%%%%%%%%%%%%%%%%%%%%%

\newcommand{\references}{%
    \cleardoublepage
    \singlespacing
    \phantomsection
    \addcontentsline{toc}{chapter}{\texorpdfstring{\MakeTextUppercase{\bibname}}{\bibname}}
    \printbibliography
}
%%%%%%%%%%%%%%%%%%%%%
% Front/Main/Back Separators
%%%%%%%%%%%%%%%%%%%%%

\newcommand{\preliminary}{%
    \pagenumbering{roman}
    \setcounter{tocdepth}{1} % Sets the ToC depth to chapters and above
    \setcounter{secnumdepth}{1} % Sets the section numbering depth to chapters and above
}

\newcommand{\mainmatter}{%
    \cleardoublepage
    \doublespacing
    \pagenumbering{arabic}
    \setcounter{tocdepth}{\value{secnumdefault}}
    \setcounter{secnumdepth}{\value{secnumdefault}}
}

%%% ------------------------------------------------------------------
%           Module 8: Depreciated Function Errors
%%% -------------------------------------------------------------------

\newcommand{\libraryrights}{%
    \ClassError{maine-thesis}{The Graduate School no longer requires\MessageBreak a Library Rights Statement page}{Please remove \protect\libraryrights\space from your thesis.}%
  }

\newcommand{\dissacceptance}{%
    \ClassError{maine-thesis}{The Graduate School no longer requires\MessageBreak a Dissertation Acceptance page}{Please remove \protect\dissacceptance\space from your thesis.}%
}
%</class>

\endinput
%%
%% End of file `maine-thesis.cls'